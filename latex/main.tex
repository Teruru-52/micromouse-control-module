\documentclass[a5paper]{ltjsarticle}
\usepackage{amsmath}
\title{加減速軌道設計}
\author{Ryotaro Onuki (kerikun11+github@gmail.com)}
\date{\today}
\begin{document}
\maketitle

\section{速度拘束}

\begin{align}
    j_m & = \mathrm{sign}(v_e-v_s) \times|j_{\max}| \\
    a_m & = \mathrm{sign}(v_e-v_s) \times|a_{\max}|
\end{align}

\begin{align}
    j(t)
     & :=
    \left\{ \begin{array}{ll}
        0    & (\hspace{2.2em}t \le t_0) \\
        j_m  & (t_0 < t \le t_1)         \\
        0    & (t_1 < t \le t_2)         \\
        -j_m & (t_2 < t \le t_3)         \\
        0    & (t_3 < t \hspace{2.2em})
    \end{array} \right.
    \\
    a(t)
     & :=
    \left\{ \begin{array}{ll}
        0           & (\hspace{2.2em}t \le t_0) \\
        j_m(t-t_0)  & (t_0 < t \le t_1)         \\
        a_m         & (t_1 < t \le t_2)         \\
        -j_m(t-t_3) & (t_2 < t \le t_3)         \\
        0           & (t_3 < t \hspace{2.2em})
    \end{array} \right.
    \\
    v(t)
     & :=
    \left\{ \begin{array}{ll}
        v_0                           & (\hspace{2.2em} t \le t_0) \\
        v_0 + \frac{1}{2}j_m(t-t_0)^2 & (t_0 < t \le t_1)          \\
        v_1 + a_m(t-t_1)              & (t_1 < t \le t_2)          \\
        v_3 - \frac{1}{2}j_m(t-t_3)^2 & (t_2 < t \le t_3)          \\
        v_3                           & (t_3 < t \hspace{2.2em})
    \end{array} \right.
    \\
    x(t)
     & :=
    \left\{ \begin{array}{ll}
        x_0 + v_0(t-t_0)                           & (\hspace{2.2em} t \le t_0) \\
        x_0 + v_0(t-t_0) + \frac{1}{6}j_m(t-t_0)^3 & (t_0 < t \le t_1)          \\
        x_1 + v_1(t-t_1) + \frac{1}{2}a_m(t-t_1)^2 & (t_1 < t \le t_2)          \\
        x_3 + v_3(t-t_3) - \frac{1}{6}j_m(t-t_3)^3 & (t_2 < t \le t_3)          \\
        x_3 + v_3(t-t_3)                           & (t_3 < t \hspace{2.2em})
    \end{array} \right.
\end{align}

$$
    \left\{ \begin{array}{l}
        \begin{array}{ll}
            \left\{ \begin{array}{l}
                t_0 := 0         \\
                t_1 := t_0 + t_c \\
                t_2 := t_1 + t_m \\
                t_3 := t_2 + t_c
            \end{array} \right.
             &
            (t_m > 0)
            \\
            \left\{ \begin{array}{l}
                t_0 := 0                                   \\
                t_1 := t_0 + \sqrt{\frac{1}{j_m}(v_e-v_s)} \\
                t_2 := t_1                                 \\
                t_3 := t_2 + (t_1-t_0)
            \end{array} \right.
             &
            (t_m \le 0)
        \end{array}
    \end{array} \right.
$$

\begin{align}
    t_m & := \frac{1}{a_m}(v_e - v_s) - \frac{a_m}{j_m} \\
    t_c & := \frac{a_m}{j_m}
\end{align}

$$
    \left\{ \begin{array}{l}
        v_0 := v_s    \\
        v_1 := v(t_1) \\
        v_2 := v(t_2) \\
        v_3 := v_e
    \end{array} \right.
    \quad
    \left\{ \begin{array}{l}
        x_0 := 0      \\
        x_1 := x(t_1) \\
        x_2 := x(t_2) \\
        x_3 := x_0 + \frac{1}{2}(v_0 + v_3)(t_3-t_0)
    \end{array} \right.
$$

\section{距離拘束}
始点速度$v_s$ と目標速度 $v_t$ の速度差が小さいとき,
終点速度 $v_e$ を決定する.
$$
    v_e(j_m, a_m, v_s, v_t, d) :=
    \left\{\begin{array}{ll}
        v_{e1} & (d > d_{t_m=0} \mbox{\quad and \quad} v_{t_m=0} > 0)
        \\
        v_{e2} & (\mbox{otherwise})
    \end{array}\right.
$$
ただし,
$$
    \begin{array}{l@{~}l}
        d_{t_m=0} & := \left( v_s + \frac{1}{2} a_m t_c \right) t_c
        \\
        v_{t_m=0} & := \frac{d}{t_c} - v_s
        \\
        t_c       & := \frac{a_m}{j_m}
        \\
        v_{e1}    & :=
        \frac{-a_m t_c + \sqrt{a_m^2 t_c^2-4(a_m t_c v_s - v_s^2 - 2a_m d)}}{2}
        \\
        v_{e2}    & :=
        \frac{1}{3}\left(c +\frac{4a^2}{c}
        -a
        \right)
        \\
        a         & := v_s
        \\
        b         & :=                       \frac{a_md^2}{t_c}
        \\
        c         & :=                       \sqrt[3]{\frac{\sqrt{27b(32a^3+27b)} + 16a^3+27b}{2}}
    \end{array}
$$

走行距離に余裕があるとき,
最大速度 $v_m$ を決定する.
$$
    v_m(t_c, a_m, v_s, v_a, v_e, d) := \min\{v_a,~v_{m2}\}
$$
ただし,
$$
    \begin{array}{l@{~}l}
        v_{m2} & := \max\{v_s, v_e, v_{m1}\}                                                   \\
        v_{m1} & := \frac{-a_mt_c + \sqrt{a_m^2t_c^2-(v_s+v_e)a_mt_c+4a_md+2(v_s^2+v_e^2)}}{2}
    \end{array}
$$

$$
    \begin{array}{l@{~}l}
        t_0 & := 0
        \\
        t_1 & := ac.t_{end}
        \\
        t_2 & := t_1 +             \frac{d - ac.x_{end} - dc.x_{end}}{v_m}
        \\
        t_3 & := t_2 + dc.t_{end}
    \end{array}
$$

\subsection{導出}

$$
    \begin{array}{l@{~}l}
        d_m & =                  \int_{t_0}^{t_1}v(t) dt + \int_{t_1}^{t_2}v(t) dt
        \\
            & =                  \int_{t_0}^{t_1}\left( v_0+\frac{1}{2} j_m(t-t_0)^2 \right) dt
        \\
            & \quad+ \int_{t_1}^{t_2}\left(
        v_0+                     \frac{1}{2} j_m(t_1-t_0)^2
        +j_m(t_1-t_0)(t-t_1) -   \frac{1}{2}j_m(t-t_1)^2
        \right) dt
        \\
            & = 2v_0t_c+a_mt_c^2
    \end{array}
$$

ただし,$j_m:=a_m/t_c$

$$
    \begin{array}{l@{~}l}
        d                      & = \frac{1}{2}(v_s+v_{e1})(t_3-t_0)
        \\
                               & = \frac{1}{2}(v_s+v_{e1})(t_c+t_m+t_c)
        \\
                               & = \frac{1}{2}(v_s+v_{e1})\left(t_c+\frac{v_{e1}-v_s}{a_m}\right)
        \\
        \Leftrightarrow v_{e1} & =
        \frac{1}{2}\left(\sqrt{4v_s^2-4v_sa_mt_c+a_m(a_mt_c^2+8d)}-a_mt_c\right)
    \end{array}
$$

$$
    \begin{array}{l@{~}l}
        d        & =                                                         \frac{1}{2}(v_s+v_{e2})(t_3-t_0)
        \\
                 & =                                                         \frac{1}{2}(v_s+v_{e2})2\sqrt{\frac{t_c}{a_m}(v_{e2}-v_s)}
        \\
        \Leftrightarrow
        v_{e2}^3 & + v_s v_{e2}^2-v_s^2v_{e2}-v_s^3-\frac{a_md^2}{t_c} = 0
        \\
        \Leftrightarrow
        v_{e2}   & =
        \frac{1}{3}\left(c +\frac{4a^2}{c}
        -a
        \right)
        \\
        a        & := v_s
        \\
        b        & :=                                                        \frac{a_md^2}{t_c}
        \\
        c        & :=                                                        \sqrt[3]{\frac{\sqrt{27b(32a^3+27b)} + 16a^3+27b}{2}}
    \end{array}
$$

以上をまとめると,たどり着き得る終点速度$v_{e1}$は,

$$
    v_{e} :=
    \left\{\begin{array}{ll}
        v_{e1} & (d        \ge d_m) \\
        v_{e2} & (d < d_m)
    \end{array}\right.
$$

$$
    \begin{array}{l@{~}l}
        d                      & =
        \frac{1}{2}(v_s+v_{m1})(t_1-t_0)+
        \frac{1}{2}(v_{m1}+v_e)(t_3-t_2)
        \\
                               & =
        \frac{1}{2}(v_s+v_{m1})\left(t_c+\frac{v_{m1}-v_s}{a_m}\right)+
        \frac{1}{2}(v_{m1}+v_e)\left(t_c+\frac{v_e-v_{m1}}{a_m}\right)
        \\
        \Leftrightarrow v_{m1} & := \frac{-a_mt_c + \sqrt{a_m^2t_c^2-(v_s+v_e)a_mt_c+4a_md+2(v_s^2+v_e^2)}}{2}
    \end{array}
$$

$$
    v_{m2} := \max\{v_s,~v_e,~v_{m1}\}
$$

$$
    v_m = \min\{v_a,~v_{m2}\}
$$

\end{document}
