%%============================================================================%%
\documentclass[a5paper]{ltjsarticle}
\usepackage[margin=15mm]{geometry}
\usepackage{amsmath}
\usepackage{ascmac} % for \itembox
\usepackage{amsfonts} % for \mathbb
%%============================================================================%%
\title{曲線加速の設計(書き途中)}
\author{Ryotaro Onuki (kerikun11+github@gmail.com)}
\date{\today}
\pagestyle{headings}
\setlength\fullwidth\textwidth % to fit header size
\allowdisplaybreaks
%%============================================================================%%
\begin{document}
\maketitle
%%============================================================================%%
\section{概要}
曲線加速の設計は次の2つの段階に分かれている.
\begin{itembox}[l]{設計1}
    \quad
    始点速度から終点速度までの,
    最大躍度,最大加速度の拘束を満たす速度軌道を設計する.
    \begin{itemize}
        \item 飽和: $j_{\max},~ a_{\max} \in \mathbb{R}_+$
        \item 速度: $v_\mathrm{start},~ v_\mathrm{end} \in \mathbb{R}$
        \item 出力: $j(t),~ a(t),~ v(t),~ x(t),~ \forall t \in \mathbb{R}$
    \end{itemize}
\end{itembox}
\begin{itembox}[l]{設計2}
    \quad
    始点速度から目標速度までの,
    最大躍度,最大加速度,最大速度,移動距離の拘束を満たす速度軌道を設計する.
    \begin{itemize}
        \item 飽和: $j_{\max},~ a_{\max},~ v_{\max} \in \mathbb{R}_+$
        \item 速度: $v_\mathrm{start},~ v_\mathrm{target} \in \mathbb{R}$
        \item 変位: $d \in \mathbb{R}$
        \item 初期値: $v_\mathrm{start},~ t_\mathrm{start} \in \mathbb{R}$
        \item 出力: $j(t),~ a(t),~ v(t),~ x(t),~ \forall t \in \mathbb{R}$
    \end{itemize}
\end{itembox}
%%============================================================================%%
\section{移動距離拘束なし}
\subsection{導出結果}
時刻 $t$ における,躍度 $j(t)$, 加速度 $a(t)$, 速度 $v(t)$, 位置 $x(t)$
\begin{align}
    j(t)
     & :=
    \left\{ \begin{array}{ll}
        0    & (\hspace{2.2em}t \le t_0) \\
        j_m  & (t_0 < t \le t_1)         \\
        0    & (t_1 < t \le t_2)         \\
        -j_m & (t_2 < t \le t_3)         \\
        0    & (t_3 < t \hspace{2.2em})
    \end{array} \right.
    \\
    a(t)
     & :=
    \left\{ \begin{array}{ll}
        0           & (\hspace{2.2em}t \le t_0) \\
        j_m(t-t_0)  & (t_0 < t \le t_1)         \\
        a_m         & (t_1 < t \le t_2)         \\
        -j_m(t-t_3) & (t_2 < t \le t_3)         \\
        0           & (t_3 < t \hspace{2.2em})
    \end{array} \right.
    \\
    v(t)
     & :=
    \left\{ \begin{array}{ll}
        v_0                           & (\hspace{2.2em} t \le t_0) \\
        v_0 + \frac{1}{2}j_m(t-t_0)^2 & (t_0 < t \le t_1)          \\
        v_1 + a_m(t-t_1)              & (t_1 < t \le t_2)          \\
        v_3 - \frac{1}{2}j_m(t-t_3)^2 & (t_2 < t \le t_3)          \\
        v_3                           & (t_3 < t \hspace{2.2em})
    \end{array} \right.
    \\
    x(t)
     & :=
    \left\{ \begin{array}{ll}
        x_0 + v_0(t-t_0)                           & (\hspace{2.2em} t \le t_0) \\
        x_0 + v_0(t-t_0) + \frac{1}{6}j_m(t-t_0)^3 & (t_0 < t \le t_1)          \\
        x_1 + v_1(t-t_1) + \frac{1}{2}a_m(t-t_1)^2 & (t_1 < t \le t_2)          \\
        x_3 + v_3(t-t_3) - \frac{1}{6}j_m(t-t_3)^3 & (t_2 < t \le t_3)          \\
        x_3 + v_3(t-t_3)                           & (t_3 < t \hspace{2.2em})
    \end{array} \right.
\end{align}
躍度定数 $j_m$,
加速度定数 $a_m$
\begin{align}
    j_m & = \mathrm{sign}(v_e-v_s) \times|j_{\max}| \\
    a_m & = \mathrm{sign}(v_e-v_s) \times|a_{\max}|
\end{align}
等加速度時間 $t_m$,
曲線速度時間$(t_m>0)$ $t_c$,
曲線速度時間$(t_m\leq 0)$ $t_c'$
\begin{align}
    t_m  & := \frac{1}{a_m}(v_e - v_s) - \frac{a_m}{j_m} \\
    t_c  & := \frac{a_m}{j_m}                            \\
    t_c' & := \sqrt{\frac{1}{j_m}(v_e-v_s)}
\end{align}
時刻定数 $t_0,~t_1,~t_2,~t_3$
\begin{align}
    \begin{array}{ll}
        \left\{ \begin{array}{l}
            t_0 := 0         \\
            t_1 := t_0 + t_c \\
            t_2 := t_1 + t_m \\
            t_3 := t_2 + t_c
        \end{array} \right.
         &
        (t_m > 0)
        \\
        \left\{ \begin{array}{l}
            t_0 := 0          \\
            t_1 := t_0 + t_c' \\
            t_2 := t_1        \\
            t_3 := t_2 + t_c'
        \end{array} \right.
         &
        (t_m \leq 0)
    \end{array}
\end{align}
速度定数 $v_0,~v_1,~v_2,~v_3$
\begin{align}
    \begin{array}{ll}
        \left\{ \begin{array}{l}
            v_0 := v_s                        \\
            v_1 := v_0 + \frac{1}{2}j_m t_c^2 \\
            v_2 := v_1 + a_m t_m              \\
            v_3 := v_e
        \end{array} \right.
         &
        (t_m > 0)
        \\
        \left\{ \begin{array}{l}
            v_0 := v_s                                     \\
            v_1 := v_0 + \frac{1}{2}\left( v_s+v_e \right) \\
            v_2 := v_1                                     \\
            v_3 := v_e
        \end{array} \right.
         &
        (t_m \leq 0)
    \end{array}
\end{align}
位置定数 $x_0,~x_1,~x_2,~x_3$
\begin{align}
    \begin{array}{ll}
        \left\{ \begin{array}{l}
            x_0 := 0                         \\
            x_1 := x_0 + v_0 t_c + j_m t_c^3 \\
            x_2 := x_1 + v_1 t_m             \\
            x_3 := x_0 + \frac{1}{2} (v_0+v_3) (2t_c+t_m)
        \end{array} \right.
         &
        (t_m > 0)
        \\
        \left\{ \begin{array}{l}
            x_0 := 0                                       \\
            x_1 := x_0 + v_1 t_c' + \frac{1}{6} j_m t_c'^3 \\
            x_2 := x_1                                     \\
            x_3 := x_0 + 2 v_1 t_c'
        \end{array} \right.
         &
        (t_m \leq 0)
    \end{array}
\end{align}

%%============================================================================%%
\subsection{設計手順}
\begin{enumerate}
    \item 躍度定数,加速度定数 $j_m,~a_m$ を求める.
    \item 時間定数 $t_c,~t_m$ を求める.
    \item 時刻定数 $t_i$ を求める.
    \item 速度定数 $v_i$ を求める.
    \item 位置定数 $x_i$ を求める.
    \item 関数 $j(t),~ a(t),~ v(t),~ x(t)$ を求める.
\end{enumerate}

%%============================================================================%%
\subsection{導出過程}
加速と減速の双方で共通の式を使用できるように,
躍度定数 $j_m \in \mathbb{R}$ および 加速度定数 $a_m \in \mathbb{R}$ を
\begin{align}
    j_m & = \mathrm{sign}(v_\mathrm{end}-v_\mathrm{start}) \times|j_{\max}| \\
    a_m & = \mathrm{sign}(v_\mathrm{end}-v_\mathrm{start}) \times|a_{\max}|
\end{align}
と定義する.

これらを用いて,
時刻 $t$ における,躍度 $j(t)$, 加速度 $a(t)$, 速度 $v(t)$, 位置 $x(t)$
\begin{align}
    j(t)
     & :=
    \left\{ \begin{array}{ll}
        0    & (\hspace{2.2em}t \le t_0) \\
        j_m  & (t_0 < t \le t_1)         \\
        0    & (t_1 < t \le t_2)         \\
        -j_m & (t_2 < t \le t_3)         \\
        0    & (t_3 < t \hspace{2.2em})
    \end{array} \right.
    \\
    a(t)
     & :=
    \left\{ \begin{array}{ll}
        0           & (\hspace{2.2em}t \le t_0) \\
        j_m(t-t_0)  & (t_0 < t \le t_1)         \\
        a_m         & (t_1 < t \le t_2)         \\
        -j_m(t-t_3) & (t_2 < t \le t_3)         \\
        0           & (t_3 < t \hspace{2.2em})
    \end{array} \right.
    \\
    v(t)
     & :=
    \left\{ \begin{array}{ll}
        v_0                           & (\hspace{2.2em} t \le t_0) \\
        v_0 + \frac{1}{2}j_m(t-t_0)^2 & (t_0 < t \le t_1)          \\
        v_1 + a_m(t-t_1)              & (t_1 < t \le t_2)          \\
        v_3 - \frac{1}{2}j_m(t-t_3)^2 & (t_2 < t \le t_3)          \\
        v_3                           & (t_3 < t \hspace{2.2em})
    \end{array} \right.
    \\
    x(t)
     & :=
    \left\{ \begin{array}{ll}
        x_0 + v_0(t-t_0)                           & (\hspace{2.2em} t \le t_0) \\
        x_0 + v_0(t-t_0) + \frac{1}{6}j_m(t-t_0)^3 & (t_0 < t \le t_1)          \\
        x_1 + v_1(t-t_1) + \frac{1}{2}a_m(t-t_1)^2 & (t_1 < t \le t_2)          \\
        x_3 + v_3(t-t_3) - \frac{1}{6}j_m(t-t_3)^3 & (t_2 < t \le t_3)          \\
        x_3 + v_3(t-t_3)                           & (t_3 < t \hspace{2.2em})
    \end{array} \right.
\end{align}
を考える.
ただし,$j(t)$を与えて順に積分を行った.
以下で各定数 $t_i,~ v_i,~ x_i$ を求める.

ここで,
加速中に等加速度直線運動が存在する場合と,存在しない場合があることに注意する.

曲線速度の時間 $t_c$ は,躍度 $j_m$ で加速度 $a_m$ になる時間なので,
\begin{align}
    t_c = \frac{a_m}{j_m}
\end{align}
と求めることができる.
次に,
等加速度直線運動の時間 $t_m$ を求める.
まず,
$v_s$ と $v_e$ の速度差が十分にあり, $t_m>0$ と仮定する.
加速度 $a(t)$ を各時刻区間において積分
\begin{align}
     &
    v_e
    =
    v_s + \int_{0}^{t_c}j_m t dt + \int_{t_c}^{t_c+t_m} a_m dt + \int_{t_c+t_m}^{t_c+t_m+t_c} (-j_m)(t-t_c-t_m) dt
    \\
     &
    \Leftrightarrow\quad
    t_m = \frac{1}{a_m}(v_e-v_s) - t_c
    = \frac{1}{a_m}(v_e-v_s) - \frac{a_m}{j_m}
\end{align}
をして,
等加速度直線運動の時間 $t_m$ を求めることができる.
したがって,
\begin{align}
    \begin{array}{ll}
        \left\{ \begin{array}{l}
            t_0 := 0         \\
            t_1 := t_0 + t_c \\
            t_2 := t_1 + t_m \\
            t_3 := t_2 + t_c
        \end{array} \right.
         &
        (t_m > 0)
        \\
        \left\{ \begin{array}{l}
            v_0 := v_s                        \\
            v_1 := v_0 + \frac{1}{2}j_m t_c^2 \\
            v_2 := v_1 + a_m t_m              \\
            v_3 := v_e
        \end{array} \right.
         &
        (t_m > 0)
        \\
        \left\{ \begin{array}{l}
            x_0 := 0                         \\
            x_1 := x_0 + v_0 t_c + j_m t_c^3 \\
            x_2 := x_1 + v_1 t_m             \\
            x_3 := x_0 + \frac{1}{2} (v_0+v_3) (2t_c+t_m)
        \end{array} \right.
         &
        (t_m > 0)
    \end{array}
\end{align}
が得られる.
ただし,
速度曲線の対称性より,走行距離 $x_3-x_0$ は,$x$軸と始点速度 $v_s$, 終点速度 $v_e$ からなる台形の面積に等しい.
したがって,
\begin{align}
    x_3-x_0 = \frac{1}{2}\left( v_e-v_s \right) \left( t_3-t_0 \right)
\end{align}
と求めることができる.

逆に, $t_m \leq 0$ となるとき,つまり,
最大躍度 $j_m$, 最大加速度 $a_m$, 始点速度 $v_s$, 終点速度 $v_e$ が
\begin{align}
     &
    t_m = \frac{1}{a_m}(v_e-v_s) - \frac{a_m}{j_m} < 0
    \\
     &
    \Leftrightarrow
    \quad
    \left\{
    \begin{array}{ll}
        v_e < v_s + \frac{a_m^2}{j_m} & (a_m \ge 0)
        \\
        v_e > v_s + \frac{a_m^2}{j_m} & (a_m <0)
    \end{array}
    \right.
\end{align}
の関係のとき,等加速度直線運動は存在しないことがわかる.
このとき,
新たな曲線加速の時間 $t_c',~(0 < t_c' < t_c)$ を求める.
始点速度 $v_s$ と 終点速度 $v_e$ および新たな曲線加速の時間 $t_c'$ の関係より,
\begin{align}
     &
    v_e = v_s + \int_{0}^{t_c'} j_m t dt + \int_{t_c'}^{t_c'+t_c'} (-j_m) (t-t_c') dt
    \\
     &
    \Leftrightarrow\quad
    t_c' = \sqrt{\frac{1}{j_m}(v_e-v_s)}
\end{align}
と求めることができる.
したがって,
\begin{align}
    \begin{array}{ll}
        \left\{ \begin{array}{l}
            t_0 := 0          \\
            t_1 := t_0 + t_c' \\
            t_2 := t_1        \\
            t_3 := t_2 + t_c'
        \end{array} \right.
         &
        (t_m \leq 0)
        \\
        \left\{ \begin{array}{l}
            v_0 := v_s                                     \\
            v_1 := v_0 + \frac{1}{2}\left( v_s+v_e \right) \\
            v_2 := v_1                                     \\
            v_3 := v_e
        \end{array} \right.
         &
        (t_m \leq 0)
        \\
        \left\{ \begin{array}{l}
            x_0 := 0                                       \\
            x_1 := x_0 + v_1 t_c' + \frac{1}{6} j_m t_c'^3 \\
            x_2 := x_1                                     \\
            x_3 := x_0 + 2 v_1 t_c'
        \end{array} \right.
         &
        (t_m \leq 0)
    \end{array}
\end{align}
が得られる.

\clearpage
%%============================================================================%%
\section{移動距離拘束つき}
\begin{table}[htbp]
    \centering
    \caption{変数のまとめ}
    \begin{tabular}{c|c|c|c|c}
        パラメータ & 記号  & 単位        & 属性   & 備考             \\ \hline\hline
        最大躍度   & $j_m$ & [m/s${}^3$] & 拘束   &                  \\
        最大加速度 & $a_m$ & [m/s${}^2$] & 拘束   &                  \\
        最大速度   & $v_m$ & [m/s]       & 拘束   &                  \\
        始点速度   & $v_s$ & [m/s]       & 初期値 &                  \\
        目標速度   & $v_t$ & [m/s]       & 目標   & 目標終点速度     \\
        終点速度   & $v_e$ & [m/s]       & 結果   & 到達可能終点速度 \\
        移動距離   & $d$   & [m]         & 拘束   &                  \\
        始点時刻   & $t_s$ & [m]         & 初期値 & オプション       \\
        始点位置   & $x_s$ & [m]         & 初期値 & オプション       \\
    \end{tabular}
\end{table}

\subsection{導出結果}
始点速度 $v_s$ と目標速度 $v_t$ の速度差が小さいとき,
走行距離 $d$ の拘束により
目標速度 $v_t$ に達しない場合がある.
それを考慮して終点速度 $v_e$ を決定する.

終点速度 $v_e$
\begin{align}
    v_e(j_m,a_m,v_s,v_t,d) & :=
    \left\{
    \begin{array}{lll}
        v_t    & \mbox{if\quad}      & \|d\| \geq d_{v_s,v_t}
        \\
        v_{e1} & \mbox{else if\quad} & \|d\| > \|d_{t_m=0}\| \mbox{\quad and \quad} d\, v_{t_m=0} > 0
        \\
        v_{e2} & \mbox{else}
    \end{array}
    \right.
\end{align}
始点速度 $v_s$ から目標速度 $v_t$ まで変化するのに必要な最小移動距離 $d_{v_s,v_t}$
\begin{align}
    d_{v_s,v_t}(j_m,a_m,v_s,v_t) & := \frac{1}{2}(v_s+v_t) \, t_{v_s,v_t}        \\
    t_{v_s,v_t}                  & :=
    \left\{\begin{array}{ll}
        2t_c + t_m & (t_m>0)    \\
        2t_c'      & (t_m\leq0)
    \end{array}\right.
    \\
    t_m                          & := \frac{1}{a_m}(v_t - v_s) - \frac{a_m}{j_m} \\
    t_c                          & := \frac{a_m}{j_m}                            \\
    t_c'                         & := \sqrt{\frac{1}{j_m}(v_t-v_s)}
\end{align}
等加速度時間 $t_m = 0$となるときの移動距離 $d_{t_m=0}$ ,終点速度 $v_{t_m=0}$
\begin{align}
    d_{t_m=0} & := \left( v_s + \frac{1}{2}a_m t_c \right) t_c
    \\
    v_{t_m=0} & := \frac{j_m}{a_m} d - v_s
\end{align}
終点速度
\begin{align}
    v_{e1}  & := \frac{-a_m t_c + \sqrt{a_m^2 t_c^2-4(a_m t_c v_s - v_s^2 - 2a_m d)}}{2}
    \\
    v_{e2}  & := c +\bar{c} - \frac{1}{3} a = c + \frac{4a^2}{9c} - \frac{1}{3} a
    \\
    a       & := v_s
    \\
    b       & := j_m d^2
    \\
    c       & := \sqrt[3]{8a^3/27+b/2 + \|b\|\sqrt{8a^3/b/27+1/4}}
    \\
    \bar{c} & := \sqrt[3]{8a^3/27+b/2 - \|b\|\sqrt{8a^3/b/27+1/4}}
\end{align}

始点速度 $v_s$ と終点速度 $v_e$ ,移動変位 $d$ を満たす
到達可能速度 $v_r$
\begin{align}
    v_r := \frac{-a_mt_c \pm \sqrt{a_m^2t_c^2-(v_s+v_e)a_mt_c+4a_md+2(v_s^2+v_e^2)}}{2}
\end{align}

飽和速度
$$
    v_m = \min\{v_r,~v_{\max} \}
$$

\clearpage
\begin{align}
    v_e(j_m, a_m, v_s, v_t, d) :=
    \left\{\begin{array}{ll}
        v_{e1} & (d > d_{t_m=0} \mbox{\quad and \quad} v_{t_m=0} > 0)
        \\
        v_{e2} & (\mbox{otherwise})
    \end{array}\right.
\end{align}
ただし,
\begin{align}
    d_{t_m=0} & := \left( v_s + \frac{1}{2} a_m t_c \right) t_c
    \\
    v_{t_m=0} & := \frac{d}{t_c} - v_s
    \\
    t_c       & := \frac{a_m}{j_m}
    \\
    v_{e1}    & := \frac{-a_m t_c + \sqrt{a_m^2 t_c^2-4(a_m t_c v_s - v_s^2 - 2a_m d)}}{2}
    \\
    v_{e2}    & := \frac{1}{3}\left(c +\frac{4a^2}{c} -a \right)
    \\
    a         & := v_s
    \\
    b         & := \frac{a_md^2}{t_c}
    \\
    c         & := \sqrt[3]{\frac{\sqrt{27b(32a^3+27b)} + 16a^3+27b}{2}}
\end{align}
簡略化
\begin{align}
    v_{e2}  & := c +\bar{c} - a/3 = c + 4a^2/c/9 - a/3
    \\
    a       & := v_s
    \\
    b       & := \frac{a_md^2}{t_c}
    \\
    c       & := \sqrt[3]{8a^3/27+b/2 + \|b\|\sqrt{8a^3/b/27+1/4)}}
    \\
    \bar{c} & := \sqrt[3]{8a^3/27+b/2 - \|b\|\sqrt{8a^3/b/27+1/4)}}
\end{align}

走行距離に余裕があるとき,
最大速度 $v_m$ を決定する.
$$
    v_m(t_c, a_m, v_s, v_a, v_e, d) := \min\{v_a,~v_{m2}\}
$$
ただし,
$$
    \begin{array}{l@{~}l}
        v_{m2} & := \max\{v_s, v_e, v_{m1}\}                                                   \\
        v_{m1} & := \frac{-a_mt_c + \sqrt{a_m^2t_c^2-(v_s+v_e)a_mt_c+4a_md+2(v_s^2+v_e^2)}}{2}
    \end{array}
$$

$$
    \begin{array}{l@{~}l}
        t_0 & := 0
        \\
        t_1 & := ac.t_{end}
        \\
        t_2 & := t_1 +             \frac{d - ac.x_{end} - dc.x_{end}}{v_m}
        \\
        t_3 & := t_2 + dc.t_{end}
    \end{array}
$$

\subsection{導出}

$$
    \begin{array}{l@{~}l}
        d_m & =                  \int_{t_0}^{t_1}v(t) dt + \int_{t_1}^{t_2}v(t) dt
        \\
            & =                  \int_{t_0}^{t_1}\left( v_0+\frac{1}{2} j_m(t-t_0)^2 \right) dt
        \\
            & \quad+ \int_{t_1}^{t_2}\left(
        v_0+                     \frac{1}{2} j_m(t_1-t_0)^2
        +j_m(t_1-t_0)(t-t_1) -   \frac{1}{2}j_m(t-t_1)^2
        \right) dt
        \\
            & = 2v_0t_c+a_mt_c^2
    \end{array}
$$

ただし,$j_m:=a_m/t_c$

$$
    \begin{array}{l@{~}l}
        d                      & = \frac{1}{2}(v_s+v_{e1})(t_3-t_0)
        \\
                               & = \frac{1}{2}(v_s+v_{e1})(t_c+t_m+t_c)
        \\
                               & = \frac{1}{2}(v_s+v_{e1})\left(t_c+\frac{v_{e1}-v_s}{a_m}\right)
        \\
        \Leftrightarrow v_{e1} & =
        \frac{1}{2}\left(\sqrt{4v_s^2-4v_sa_mt_c+a_m(a_mt_c^2+8d)}-a_mt_c\right)
    \end{array}
$$

$$
    \begin{array}{l@{~}l}
        d        & =                                                         \frac{1}{2}(v_s+v_{e2})(t_3-t_0)
        \\
                 & =                                                         \frac{1}{2}(v_s+v_{e2})2\sqrt{\frac{t_c}{a_m}(v_{e2}-v_s)}
        \\
        \Leftrightarrow
        v_{e2}^3 & + v_s v_{e2}^2-v_s^2v_{e2}-v_s^3-\frac{a_md^2}{t_c} = 0
        \\
        \Leftrightarrow
        v_{e2}   & =
        \frac{1}{3}\left(c +\frac{4a^2}{c}
        -a
        \right)
        \\
        a        & := v_s
        \\
        b        & :=                                                        \frac{a_md^2}{t_c}
        \\
        c        & :=                                                        \sqrt[3]{\frac{\sqrt{27b(32a^3+27b)} + 16a^3+27b}{2}}
    \end{array}
$$

以上をまとめると,たどり着き得る終点速度$v_{e1}$は,

$$
    v_{e} :=
    \left\{\begin{array}{ll}
        v_{e1} & (d        \ge d_m) \\
        v_{e2} & (d < d_m)
    \end{array}\right.
$$

始点速度 $v_s$ と終点速度 $v_e$ ,移動変位 $d$ を満たす
到達可能速度 $v_r$ の導出
\begin{align}
    d               & =
    \frac{1}{2}(v_s+v_r)(t_1-t_0)+
    \frac{1}{2}(v_r+v_e)(t_3-t_2)
    \\
                    & =
    \frac{1}{2}(v_s+v_r)\left(t_c+\frac{v_r-v_s}{a_m}\right)+
    \frac{1}{2}(v_r+v_e)\left(t_c+\frac{v_e-v_r}{a_m}\right)
    \\
    \Leftrightarrow & \quad
    v_r := \frac{-a_mt_c + \sqrt{a_m^2t_c^2-(v_s+v_e)a_mt_c+4a_md+2(v_s^2+v_e^2)}}{2}
\end{align}

$$
    v_m = \min\{v_r,~v_{\max} \}
$$

\end{document}
